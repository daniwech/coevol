\documentclass[
12.5pt, 					% Main document font size
a4paper, 				% Paper type, use 'letterpaper' for US Letter paper
oneside,					% One page layout (no page indentation)
headinclude,footinclude, % Extra spacing for the header and footer
BCOR5mm, 				% Binding correction
abstract=on
]{scrreprt}

%-*-*-*-*-*-*-*-*-*-*-*-*-*-*-*-*-*-*-*-*-*-*-*-*-*-*-*-*-*-*-*-*-*-*-*-*-*-*-*-* %
%	INCLUDES                                          %
%-*-*-*-*-*-*-*-*-*-*-*-*-*-*-*-*-*-*-*-*-*-*-*-*-*-*-*-*-*-*-*-*-*-*-*-*-*-*-*-* %

\usepackage{amsmath}
\usepackage{graphicx}
\usepackage[numbers]{natbib}
\usepackage{doi}
\usepackage{color}   %May be necessary if you want to color links
\usepackage{hyperref}
\usepackage[margin=2.8cm]{geometry}

\usepackage{float}
\usepackage{chngcntr}
\counterwithout{figure}{chapter}
\usepackage{fancyhdr}
\usepackage{multirow,array}

\usepackage[font=small,labelfont=bf]{caption}
\usepackage{subcaption}
\usepackage{chngcntr}
\usepackage{mathtools}
\usepackage{enumitem}
\usepackage{tikz}
\usetikzlibrary[topaths]

\usepackage{ifthen}
\usepackage{relsize}
\usepackage{amsmath,amssymb}
\usepackage{bibentry}
\usepackage{arydshln}
\usepackage{dirtytalk}
\usepackage{tabularx}
\usepackage{booktabs}
\usetikzlibrary{decorations.pathreplacing}
\usepackage{lineno,xcolor}



%%%%%%%%%%%%%%%%%%%%%%%%%%%%%%%%%%%%%%%%%%%%%%%%%%%%%%%%%%%%%%%%%%%%%
%%%%%%%%%%%%%%%%%%%%%%%%%%%%%%%%%%%%%%%%%%%%%%%%%%%%%%%%%%%%%%%%%%%%%
% COMMANDS
%%%%%%%%%%%%%%%%%%%%%%%%%%%%%%%%%%%%%%%%%%%%%%%%%%%%%%%%%%%%%%%%%%%%%
%%%%%%%%%%%%%%%%%%%%%%%%%%%%%%%%%%%%%%%%%%%%%%%%%%%%%%%%%%%%%%%%%%%%%
\newcount\mycount

% Turn of indent at paragraph start
\setlength{\parindent}{0pt}	

\counterwithout{figure}{section}


% Bibliography
\let\oldbibliography\bibliography		
\renewcommand{\bibliography}[1]{
{
  \let\chapter\section% Copy \section over \chapter
  \oldbibliography{#1}}
}
\bibliographystyle{unsrtnat}


\renewcommand\thesection{\arabic{section}}

\hypersetup{	colorlinks,breaklinks,
            	urlcolor=[rgb]{0.07,0.27,0.40},
            	linkcolor=[rgb]{0.07,0.27,0.40},
			anchorcolor=[rgb]{0.07,0.27,0.40},
		}

\numberwithin{equation}{chapter}
\counterwithout{equation}{chapter}

% Table padding of rows
\newcommand\Tstrut{\rule{0pt}{3.6ex}}       % "top" strut
\newcommand\Bstrut{\rule[-0.9ex]{0pt}{0pt}} % "bottom" strut
\newcommand{\TBstrut}{\Tstrut\Bstrut} % top&bottom struts



\begin{document}
\nobibliography*


%%%%%%%%%%%%%%%%%%%%%%%%%%%%%%%%%%%%%%%%%%%%%%%%%%%%%%%%%%%%%%%%%%%%%
%%%%%%%%%%%%%%%%%%%%%%%%%%%%%%%%%%%%%%%%%%%%%%%%%%%%%%%%%%%%%%%%%%%%%
% MODEL DEFINITION
%%%%%%%%%%%%%%%%%%%%%%%%%%%%%%%%%%%%%%%%%%%%%%%%%%%%%%%%%%%%%%%%%%%%%
%%%%%%%%%%%%%%%%%%%%%%%%%%%%%%%%%%%%%%%%%%%%%%%%%%%%%%%%%%%%%%%%%%%%%
\chapter*{Eco-CoEvo-Model}
\vspace{1cm}

\begin{tabular}{l l p{13cm}}
$n$ & $\coloneqq$ & The number of species. \\ \\

$k$ & $\coloneqq$ & The number of traits per species. \\ \\

$N$ & $\coloneqq$ & A column vector of size $n$ denoting the abundance of each species. 
To refer to species the indices $i$ and $j$ are used. An element $N_i$ should be in the range $[0,1]$. \\ \\

$V$ & $\coloneqq$ & A column vector of size $n \times k$ indicating the trait values. To refer to traits the indices $p$, $q$ and $m$ are used. For instance $V_p^i$ refers to trait $p$ of species $i$.
To refer to only the trait vector of a single species $i$, the notation $V^i$ is used. \\\\


$B^i$ & $\coloneqq$ & A $k \times k$ matrix defining the epistatic interactions between traits of species $i$.  An entry $B_{pq}^i \in [-1,1]$ defines the effect of trait $q$ on trait $p$ in species $i$.
Currently we assume $B_{pp}=0$, because it is simpler to compute the partial derivative of the fitness function. \\ \\


$\rho$ & $\coloneqq$  &Affects the steepness of the fitness landscape. We need to explore the effect and its usefulness.\\ \\


$A^{ij}$ &  $\coloneqq$  & Defines the ecological effects of the traits of species $j$ on the traits of species $i$. An entry $A^{ij}_{pq} \in \{-1,0, 1\}$ defines the effect (positive, neutral, negative) of trait $q$ of species $j$ on trait $p$ of species $i$.
In total there are $n \times n - n $ of those matrices (two for each possible species pair).

Possible restrictions to this matrices could be:
There is only one trait involved for the interaction between two species.
In this case, there is only one non zero entry in each $A^{ij}$. \\ \\

 
$\lambda$  & $\coloneqq$ &  Separation of time scales between evolutionary and ecological dynamics.
If $\lambda < 1$ evolution is slower compared to ecological processes.
\\ \\

$\eta$ & $\coloneqq$ & Controls the ratio of the external force to the internal force in trait evolution.
\\ \\

$\gamma$ & $\coloneqq$ & Defines how much the internal fitness affects the growth rate of a species.
Its not so clear whether this parameter should be zero (what is the rational?).\\ \\

$M$ & $\coloneqq$ & The maximum of the fitness function (internal trait evolution). \\ \\

$r$ & $\coloneqq$ & Vector of size $n$, with $r_i$ being the intrinsic growth rate of species $i$.
\\ \\
$\beta$ & $\coloneqq$ & Carrying capacity (could also be per species).
\\ \\
$c^i_p$ & $\coloneqq$ & Affects the position of the global optimum of the internal fitness function of species $i$.
\\ \\
$h$ & $\coloneqq$ & Affects the maximum effect of an interaction (per trait) on the population change rate (the maximum effect is $1/h$). see functional response type II, to prevent unbounded population growth. 


\end{tabular}


\begin{table}[H]
	\captionsetup{format=plain}
    \begin{center}
    \begin{tabular}{ c : c c c : c c c : c c c }
    
    	 		& $V^1_1$	& $V^1_2$ & $V^1_3$ & $V^2_1$ & $V^2_2$ & $V^2_3$ & $V^3_1$ & $V^3_2$ & $V^3_3$\\
    
    \Bstrut $V^1_1$ 	& 0	& 0 & 0 & 0 & 0 & 0 &  0 & 0 & 0 \Tstrut\\
    \Bstrut $V^1_2$ 	& 0	& 0 & 0 & 0 & 0 & 0 &  0 & 0 & 0 \Tstrut\\
    \Bstrut $V^1_3$ 	& 0	& 0 & 0 & 0 & 0 & 0 &  0 & 0 & 0 \Tstrut\\
    	\hdashline
    \Bstrut $V^2_1$ 	& 0	& 0 & 0 & 0 & 0 & 0 &  0 & 0 & 0 \Tstrut\\
   \Bstrut  $V^2_2$ 	& 0	& 0 & 0 & 0 & 0 & 0 &  0 & 0 & 0 \Tstrut\\
    \Bstrut $V^2_3$ 	& 0	& 0 & 0 & 0 & 0 & 0 &  0 & 0 & 0 \Tstrut\\
    \hdashline
    \Bstrut $V^3_1$	& 0	& 0 & 0 & 0 & 0 & 0 &  0 & 0 & 0 \Tstrut\\
    \Bstrut $V^3_2$ 	& 0	& 0 & 0 & 0 & 0 & 0 & 0 & 0 & 0 \Tstrut\\
   \Bstrut $V^3_3$	& 0	& 0 & 0 & 0 & 0 & 0 &  0 & 0 & 0 \Tstrut\\

	\end{tabular}
	
	
  	\end{center}
  	
\end{table}



\subsection*{Remarks}

\begin{itemize}
\item The structure (or complexity) of the matrices $A^{ij}$ are a major property of the model that needs to be investigated.

\end{itemize}


\section*{Ecological dynamics}


\begin{equation}
\label{eq:population_dynamics}
\frac{dN_i}{dt} = N_i \Big(r_i - \beta N_i + \gamma \Psi(V^i) + \sum_{j=1, i \neq j}^n \frac{G(V^i, V^j) ) N_j}{1 + h N_j} \Big)
\end{equation}



\begin{equation}
\label{eq:population_dynamics_interactions}
G(V^i, V^j) = \sum_{p=1}^{k} g(V^i_p, V^j)
\end{equation}

\begin{equation}
\label{eq:population_dynamics_interactions}
g(V^i_p, V^j) = \sum_{q=1}^{k} A^{ij}_{pq} F(V^i_p, V^j_q) 
\end{equation}


The function $F:  \mathbb{R}^2 \rightarrow [0,1]$ gives the interaction strength contribution between two species due to the two traits. Currently we only consider a trait match function which has its maximum if the two traits are equal, but other function could be applied as well.
\newline
\begin{equation}
\label{eq:population_dynamics_interactions}
F(V^i_p, V^j_q) = e^{-(V^i_p - V^j_q)^2}
\end{equation}

\section*{Evolution}

The traits of the species evolve due to internal epistasis and interactions.
Both interactions with other species and epistatic relations impose selective pressure on the traits. Generally, the traits evolve according to a gradient ascent procedure. 
We take into account the gradient of the internal fitness landscape (epistasis) as well as the gradient of the per capita growth rate of a species. Hence traits evolve in a direction, of improved fitness.


\begin{equation}
\label{eq:trait_change}
\frac{dV^i_p}{dt} = 
\lambda N_i
\Big(
\sum_{j=1}^{n} N_j \frac{\partial g(V^i_p, V^j)}{\partial V^i_p} + 
\eta \frac{\partial \Psi(V^i)}{\partial V^i_p}
\Big)
\end{equation}



\subsection*{Evolution due to epistasis}
\vspace{0.5cm}

\begin{equation}
\label{eq:trait_change_interactions}
\begin{aligned}
\frac{\partial \Psi(V^i)}{\partial V^i_p} = 
2 \rho M k^{-1} \Bigg(
\frac{ \big(-V^i_p + \mu_p(V^i) \big) }
{\big(1+ \rho (V^i_p-\mu_p(V^i))^2\big)^2} +
\sum_{q=1, p \neq q}^{k}
\frac{ B^i_{qp} \big(V^i_q - \mu_q(V^i))}
{\big(1+ \rho (V^i_q-\mu_q(V^i))^2\big)^2} 
\Bigg)
\\ \\
=
2 \rho M k^{-1} \Bigg(
\frac{ \big(c^i_p-V^i_p + \sum_{q=1}^{k} B^i_{pq} V^i_q\big) }
{\big(1+ \rho (V^i_p-\mu_p(V^i))^2\big)^2} +
\sum_{q=1, p \neq q}^{k}
\frac{ B^i_{qp} \big(V^i_q - c^i_q - \sum_{m=1}^{k} B^i_{qm} V^i_m\big)}
{\big(1+ \rho (V^i_q-\mu_q(V^i))^2\big)^2}
\Bigg)
\end{aligned}
\end{equation}
\newline


Under the assumption that $B^i_{pp}=0$, otherwise the derivative is more complicated.

\subsection*{Evolution due to interactions}
\vspace{0.5cm}

\begin{equation}
\label{eq:trait_change_interactions}
\begin{aligned}
\frac{\partial g(V^i_p, V^j)}{\partial V^i_p} = 
\sum_{q=1}^{k}
A^{ij}_{pq} 
\frac{\partial F(V^i_p, V^j_q)}{\partial V^i_p}  
= 
- \sum_{q=1}^{k} A^{ij}_{pq} (V^i_p - V^j_q) e^{-(V^i_p - V^j_q)^2}  
\end{aligned}
\end{equation}




\section*{Trait epistasis - fitness landscape}

\begin{equation}
\label{eq:fitness_function}
\Psi(V^i) = \frac{1}{k}\sum_{p=1}^k \phi_p(\vec{V^i})
\end{equation}

\begin{equation}
\label{eq:fitness_function_trait_contribution}
\phi_p(V^i) = \frac{M}{1+ \rho (V^i_p-\mu_p(V^i))^2}
\end{equation}

\begin{equation}
\label{eq:fitness_function_epistatic_effect}
\mu_p(V^i) = c_p^i + \sum_{q=1}^l B_{pq}^i V^i_q
\end{equation}

\subsubsection*{Remarks}

\begin{itemize}
\item
Potentially, the parameter $M$ could be defined per species instead of using the same for all species.

\item
Through $c_p^i$ one can determine the position of the global optimum the trait space of a species. We need to explore in more detail what it does.

\item
We need to check whether under the given definitions traits can grow to infinity.
If so, it would be desirable if the trait values are restricted to a certain range.


\end{itemize}


\section{Implementations}

An R and python implementation is on \url{https://github.com/daniwech/coevol}

\section*{Research Questions / Plan}

\begin{itemize}
\item Assuming, the entries in the $A$ and $B$ matrices are chosen randomly.
What is the effect of the properties (e.g. connectance) on the dynamical properties like,
persistence (fractions of species surviving), overall strength of ecological interactions
(sum over all interactions strengths), 
\newline

\end{itemize}

\subsection{Simple 2-Species/ 2-Trait System}

There are $3^2=9$ combinations of $B$ matrices:

$$
B :
 \begin{bmatrix}
  0 & 0 \\
  0 & 0 \\
 \end{bmatrix}
 \begin{bmatrix}
  0 & 1 \\
  0 & 0 \\
 \end{bmatrix}
  \begin{bmatrix}
  0 & 0 \\
  1& 0 \\
 \end{bmatrix}
   \begin{bmatrix}
  0 & 1 \\
  1& 0 \\
 \end{bmatrix}
  \begin{bmatrix}
  0 & -1 \\
  0 & 0 \\
 \end{bmatrix}
  \begin{bmatrix}
  0 & 0 \\
  -1 & 0 \\
 \end{bmatrix}
 \begin{bmatrix}
  0 & -1 \\
  -1 & 0 \\
 \end{bmatrix}
  \begin{bmatrix}
  0 & -1 \\
  1 & 0 \\
 \end{bmatrix}
   \begin{bmatrix}
  0 & 1 \\
  -1 & 0 \\
 \end{bmatrix}
$$

There are $3^4=81$ possible $A$ matrices.

$$
A :
 \begin{bmatrix}
  1 & 0 \\
  0 & 0 \\
 \end{bmatrix}
 \begin{bmatrix}
  0 & 1 \\
  0 & 0 \\
 \end{bmatrix}
 ...
 \begin{bmatrix}
  -1 & -1 \\
  -1 & -1 \\
 \end{bmatrix}
$$

Hence in total there are $81 \times 81 \times 9 \times 9 = 531441$ combinations (there are symmetries - try to identify them in order to reduce search space).
\newline


What is the fraction of simulations for which, at a certain parameter combination a fix point is reached after a certain time?
\newline

What is the extinction probability for different parameter combinations?
\newline

Which species dominates which species (and how does it relate to the matrices)?
\newline

Is it possible to classify the results based on the $B$ and $A$ matrices (and their combinations).
What are the properties of the matrices that most significantly affect the results?
\newline

%%%%%%%%%%%%%%%%%%%%%%%%%%%%%%%%%%%%%%%%%%%%%%%%%%%%%%%%%%%%%%%%%%%%%
%%%%%%%%%%%%%%%%%%%%%%%%%%%%%%%%%%%%%%%%%%%%%%%%%%%%%%%%%%%%%%%%%%%%%
% BIBLIOGRAPHY
%%%%%%%%%%%%%%%%%%%%%%%%%%%%%%%%%%%%%%%%%%%%%%%%%%%%%%%%%%%%%%%%%%%%%
%%%%%%%%%%%%%%%%%%%%%%%%%%%%%%%%%%%%%%%%%%%%%%%%%%%%%%%%%%%%%%%%%%%%%


\renewcommand{\bibname}{}
\vspace{2cm}
\section*{References}
\bibliography{references}


\end{document}



